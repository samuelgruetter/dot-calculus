\documentclass{llncs}

\setlength{\textwidth}{13.5cm}
\advance\evensidemargin by -.65cm
\advance\oddsidemargin by -.65cm

\usepackage{fleqn}
\usepackage{listings}
\usepackage{math}
\usepackage{amsmath}
\usepackage{latexsym}
\usepackage{bcprules}
\usepackage[T1]{fontenc}

% Prooftree formatting
\usepackage{prooftree}

\usepackage[bookmarks=true]{hyperref}
\usepackage{bookmark}

\usepackage{xcolor}

\input{dot_macros.tex}

%%%%%%%%%%%%%%%%%%%%%%%%
%%%% BEGIN DOCUMENT %%%%
%%%%%%%%%%%%%%%%%%%%%%%%

\begin{document}
\thispagestyle{plain}
\pagestyle{plain}
\mainmatter

\title{[DRAFT] The exDOT Calculus}
\author{Samuel Gruetter, Nada Amin, Martin Odersky}
\institute{EPFL}

\maketitle
\sloppy
\newcommand{\lindent}{\hspace{-4mm}}

%\newcommand{\highlight}[1]{\fcolorbox{lightgray}{lightgray}{#1}}

\newcommand{\highlight}[1]{\colorbox{lightgray}{#1}}

\newcommand{\HIGHLIGHT}[1]{\colorbox{lightgray}{$#1$}}

This document presents gDOT, and the additions made by exDOT are \highlight{highlighted} in gray.

\begin{figure}
\figurebox{
%\renewcommand{\baselinestretch}{0.95}
\pdfbookmark[0]{Syntax}{syntax}
{\bf Syntax}\medskip
    
$\ba{l@{\hspace{0.2mm}}|@{\hspace{0.2mm}}l}
\ba[t]{l@{\hspace{10mm}}l}
x, y, z    & \lindent{\mbox{Variable}} \\[0.2em]
l ::=      & \lindent{\mbox{Label}} \\
\gap L     & \mbox{Type label}\\
\gap m     & \mbox{Method label}\\[0.2em]
t, u ::=      & \lindent{\mbox{Term}} \\
\gap x     & \mbox{variable} \\
\gap \trmnew x {\seq{d}} t & \mbox{new instance} \\
%\gap \highlight{$\trmcall t m u$} & \mbox{method invocation} \\[0.2em]
\gap \trmcall t m u & \mbox{method invocation} \\[0.2em]
d ::= & \lindent{\mbox{Initialization}} \\
\gap \deftyp L T        & \mbox{field init.}\\
\gap \defmtd m x T U u  & \mbox{method init.}\\[0.2em]
\Gamma ::= \seq{x \typ T} & \lindent\mbox{Environment} \\[0.2em]
s      ::= \seq{x \mapsto \{\seq{d}\}} & \lindent\mbox{Store} \\
\ea
&

\ba[t]{l@{\hspace{10mm}}l}
S,T,U,V,W ::= & \lindent\mbox{Type}\\
\gap \Top  & \mbox{top type} \\
\gap \Bot  & \mbox{bottom type} \\
\gap \{D\} & \mbox{one-member record} \\
\gap x.L & \mbox{type selection} \\
\gap T \tand T & \mbox{intersection type} \\
\gap T \tor T  & \mbox{union type} \\[0.2em]
\gap \highlight{$x\sing$}    & \mbox{singleton type} \\[0.3em]
\gap \highlight{$\exTyp x T U$} & \mbox{existential type} \\
D ::= & \lindent\mbox{Declaration} \\
\gap \Ldecl {L} S U & \mbox{abstract type decl.} \\
\gap \mdecl m S U & \mbox{method declaration}
\ea
\ea$
}
%\caption{The DOT Calculus : Syntax}\label{dot-syntax}
\end{figure}


\begin{figure}
\figurebox{
\pdfbookmark[0]{Reduction}{reduction}
{\bf Reduction}\hfill\fbox{$\reduction t s {t'} {s'}$}

\infrule[\textsc{red-call}]
{x \mapsto \ldefs{\seq{\deftyp L W}\;\seq{\defmtd m z T U u}} \in s}
{\reduction {\trmcall x {m_i} y} s {\subst y {z_i} {u_i}} s}

\infrule[\textsc{red-new}]
{z \notin \dom(s)}
{\reduction {\trmnew x {\seq{d}} t} s
            {\subst z x t} {s \envplus{z \mapsto {\subst z x {\{\seq{d}\}}}}}}

\vspace{0.5em}

\begin{multicols}{2}

\infrule[\textsc{red-call-1}]
{\reduction {t} s {t'} {s'}}
{\reduction {\trmcall t m u} s {\trmcall {t'} m u} {s'}}

\infrule[\textsc{red-call-2}]
{\reduction {u} s {u'} {s'}}
{\reduction {\trmcall x m u} s {\trmcall x m {u'}} {s'}}

\end{multicols}

}
%\caption{The DOT Calculus : Reduction}\label{dot-red}
\end{figure}

\begin{figure}
\figurebox{
{\bf Declaration intersection}\hfill\fbox{$intersect(D_1, D_2) = D_3$}

\begin{multicols}{2}

\infrule
{D_1 = (\Ldecl L {S_1} {U_1}) ~~~~~~~ D_2 = (\Ldecl L {S_2} {U_2})}
{intersect(D_1, D_2) = (\Ldecl L {S_1 \tor S_2~} {~U_1 \tand U_2})}

\infrule
{D_1 = (\mdecl m {S_1} {U_1}) ~~~~~~~ D_2 = (\mdecl m {S_2} {U_2})}
{intersect(D_1, D_2) = (\mdecl m {S_1 \tor S_2~} {~U_1 \tand U_2})}

\end{multicols}

\linesep

{\bf Declaration union}\hfill\fbox{$union(D_1, D_2) = D_3$}

\begin{multicols}{2}

\infrule
{D_1 = (\Ldecl L {S_1} {U_1}) ~~~~~~~ D_2 = (\Ldecl L {S_2} {U_2})}
{union(D_1, D_2) = (\Ldecl L {S_1 \tand S_2~} {~U_1 \tor U_2})}

\infrule
{D_1 = (\mdecl m {S_1} {U_1}) ~~~~~~~ D_2 = (\mdecl m {S_2} {U_2})}
{union(D_1, D_2) = (\mdecl m {S_1 \tand S_2~} {~U_1 \tor U_2})}

\end{multicols}

}
\end{figure}

%%%%%%%%%%%%%%%%%%%%%%%%%%%%

\begin{figure}
\figurebox{
\pdfbookmark[0]{Membership}{membership}
{\bf Membership}\hfill{\fbox{$\Gamma \ts T \ni D$}}

\begin{multicols}{2}

\infrule[\textsc{$\Bot$-$\ni$-typ}]
{}
{\Gamma \ts \Bot \ni (\Ldecl L \Top \Bot)}

\infrule[\textsc{$\Bot$-$\ni$-mtd}]
{}
{\Gamma \ts \Bot \ni (\mdecl m \Top \Bot)}

\infrule[\textsc{rcd-$\ni$}]
{}
{\Gamma \ts \{D\} \ni D}

\infrule[\textsc{sel-$\ni$}]
{(x: T) \in \Gamma \\
 \Gamma \ts T \ni (\Ldecl L S U) \\
 \Gamma \ts U \ni D}
{\Gamma \ts x.L \ni D}

\infrule[\textsc{$\tand$-$\ni$-1}]
{\Gamma \ts T_1 \ni D \\
 \Gamma \ts T_2 \hasnt label(D)}
{\Gamma \ts T_1 \tand T_2 \ni D}

\infrule[\textsc{$\tand$-$\ni$-2}]
{\Gamma \ts T_2 \ni D \\
 \Gamma \ts T_1 \hasnt label(D)}
{\Gamma \ts T_1 \tand T_2 \ni D}

\infrule[\textsc{$\tand$-$\ni$-12}]
{\Gamma \ts T_1 \ni D_1 ~~~~ \Gamma \ts T_2 \ni D_2}
{\Gamma \ts T_1 \tand T_2 \ni intersect(D_1, D_2)}

\infrule[\textsc{$\tor$-$\ni$}]
{\Gamma \ts T_1 \ni D_1 ~~~~ \Gamma \ts T_2 \ni D_2}
{\Gamma \ts T_1 \tor T_2 \ni union(D_1, D_2)}

\end{multicols}

\linesep
   
\begin{multicols}{2}[\judgement{Non-membership}{\fbox{$\Gamma \ts T \hasnt l$}}]

\infrule[\textsc{$\Top$-$\hasnt$}]
{}
{\Gamma \ts \Top \hasnt l}

\infrule[\textsc{rcd-$\hasnt$}]
{l \neq label(D)}
{\Gamma \ts \{ D \} \hasnt l}

\infrule[\textsc{sel-$\hasnt$}]
{(x: T) \in \Gamma \\
 \Gamma \ts T \ni (\Ldecl L S U) \\
 \Gamma \ts U \hasnt l}
{\Gamma \ts x.L \hasnt l}

\infrule[\textsc{$\tand$-$\hasnt$}]
{\Gamma \ts T_1 \hasnt l ~~~~ \Gamma \ts T_2 \hasnt l}
{\Gamma \ts T_1 \tand T_2 \hasnt l}

\infrule[\textsc{$\tor$-$\hasnt$-1}]
{\Gamma \ts T_1 \ni D \\
 \Gamma \ts T_2 \hasnt label(D)}
{\Gamma \ts T_1 \tor T_2 \hasnt label(D)}

\infrule[\textsc{$\tor$-$\hasnt$-2}]
{\Gamma \ts T_2 \ni D \\
 \Gamma \ts T_1 \hasnt label(D)}
{\Gamma \ts T_1 \tor T_2 \hasnt label(D)}

\infrule[\textsc{$\tor$-$\hasnt$-12}]
{\Gamma \ts T_1 \hasnt l ~~~~ \Gamma \ts T_2 \hasnt l}
{\Gamma \ts T_1 \tor T_2 \hasnt l}

\end{multicols}
}
\end{figure}

% TODO stable_typ restriction !!!

%%%%%%%%%%%%%%%%%%%%%

\begin{figure}
  \figurebox{
\pdfbookmark[0]{Well-formedness}{wf}
\begin{multicols}{2}[\judgement{Well-formed types}{\fbox{$\Gamma \ts T \wf$}}]

      \infax[\textsc{TODO}]
      {}
      
%Inductive wf_typ_impl: ctx -> fset typ -> typ -> Prop :=
%  | wf_top: forall G A,
%      wf_typ_impl G A typ_top
%  | wf_bot: forall G A,
%      wf_typ_impl G A typ_bot
%  | wf_hyp: forall G A T,
%      T \in A ->
%      wf_typ_impl G A T
%  | wf_rcd: forall G A D,
%      wf_dec_impl G (A \u \{(typ_rcd D)}) D ->
%      wf_typ_impl G A (typ_rcd D)
%  | wf_sel: forall G A x X L T U,
%      binds x X G ->
%      stable_typ X -> (* <-- important restriction *)
%      typ_has G X (dec_typ L T U) ->
%      wf_typ_impl G A X ->
%      wf_typ_impl G A T ->
%      wf_typ_impl G A U ->
%      wf_typ_impl G A (typ_sel (avar_f x) L)
%  | wf_and: forall G A T1 T2,
%      wf_typ_impl G A T1 ->
%      wf_typ_impl G A T2 ->
%      wf_typ_impl G A (typ_and T1 T2)
%  | wf_or: forall G A T1 T2,
%      wf_typ_impl G A T1 ->
%      wf_typ_impl G A T2 ->
%      wf_typ_impl G A (typ_or T1 T2)
%  | wf_self: forall G A x X,
%      binds x X G ->
%      wf_typ_impl G A X ->
%      wf_typ_impl G A (typ_self (avar_f x))
%  | wf_ex: forall L G A T U,
%      (forall x, x \notin L -> wf_typ_impl (G & x ~ (open_typ x T)) A (open_typ x T)) ->
%      (forall x, x \notin L -> wf_typ_impl (G & x ~ (open_typ x T)) A (open_typ x U)) ->
%      wf_typ_impl G A (typ_ex T U)
%  | wf_skolem: forall L G1 G2 x A S T U,
%      (forall y, y \notin L ->
%         wf_typ_impl (G1 & x ~ (open_typ y U) & y ~ (open_typ y S) & G2) A T) ->
%      fv_typ T \c (dom (G1 & x ~ typ_ex S U & G2)) -> (* instead of "y \notin fv_typ T" *)
%      wf_typ_impl (G1 & x ~ typ_ex S U & G2) A T
%with wf_dec_impl: ctx -> fset typ -> dec -> Prop :=
%  | wf_tmem: forall G A L Lo Hi,
%      wf_typ_impl G A Lo ->
%      wf_typ_impl G A Hi ->
%      wf_dec_impl G A (dec_typ L Lo Hi)
%  | wf_mtd: forall G A m U V,
%      wf_typ_impl G A U ->
%      wf_typ_impl G A V ->
%      wf_dec_impl G A (dec_mtd m U V).

    \end{multicols}

\vspace{7cm}
    \linesep

    \begin{multicols}{2}[\judgement{Well-formed declarations}{\fbox{$\Gamma \ts D \wf$}}]
      \infrule[\textsc{wf-tmem}]
      {\Gamma \ts S \wf ~~~~ \Gamma \ts U \wf}
      {\Gamma \ts \Ldecl L S U \wf}

      \infrule[\textsc{wf-mtd}]
      {\Gamma \ts S \wf ~~~~ \Gamma \ts U \wf}
      {\Gamma \ts \mdecl m S U \wf}

    \end{multicols}
  }

\end{figure}

%%%%%%%%%%%%%%%%%%%%%

\begin{figure}
  \figurebox{

\pdfbookmark[0]{Subtyping}{subtyping}
\begin{multicols}{2}[\judgement{Subtyping}{\fbox{$\Gamma \ts S \sub T$}}]

\infrule[\textsc{$\sub$-refl}]
{\Gamma \ts T \wf}
{\Gamma \ts T \sub T}

\infrule[\textsc{$\sub$-$\Top$}]
{\Gamma \ts T \wf}
{\Gamma \ts T \sub \Top}

\infrule[\textsc{$\Bot$-$\sub$}]
{\Gamma \ts T \wf}
{\Gamma \ts \Bot \sub T}

\infrule[\textsc{$\sub$-rcd}]
{\Gamma \ts D_1 \sub D_2}
{\Gamma \ts \{D_1\} \sub \{D_2\}}

\infrule[\textsc{$\sub$-sel-l}]
{(x: T) \in \Gamma ~~~~~~~~~~~  \Gamma \ts T \wf \\
 \Gamma \ts T \ni (\Ldecl L S U) ~~~ \Gamma \ts S \sub U}
{\Gamma \ts x.L \sub U}

\infrule[\textsc{$\sub$-sel-r}]
{(x: T) \in \Gamma ~~~~~~~~~~~  \Gamma \ts T \wf \\
 \Gamma \ts T \ni (\Ldecl L S U) ~~~ \Gamma \ts S \sub U}
{\Gamma \ts S \sub x.L}

%%%%

\infrule[\textsc{$\sub$-$\tand$}]
{\Gamma \ts T \sub U_1 ~~~~ \Gamma \ts T \sub U_2}
{\Gamma \ts T \sub U_1 \tand U_2}

\infrule[\textsc{$\sub$-$\tand$-1}]
{\Gamma \ts T_1 \wf ~~~~ \Gamma \ts T_2 \wf}
{\Gamma \ts T_1 \tand T_2 \sub T_1}

\infrule[\textsc{$\sub$-$\tand$-2}]
{\Gamma \ts T_1 \wf ~~~~ \Gamma \ts T_2 \wf}
{\Gamma \ts T_1 \tand T_2 \sub T_2}

\infrule[\textsc{$\sub$-$\tor$}]
{\Gamma \ts T_1 \sub U ~~~~ \Gamma \ts T_2 \sub U}
{\Gamma \ts T_1 \tor T_2 \sub U}

\infrule[\textsc{$\sub$-$\tor$-1}]
{\Gamma \ts T_1 \wf ~~~~ \Gamma \ts T_2 \wf}
{\Gamma \ts T_1 \sub T_1 \tor T_2}

\infrule[\textsc{$\sub$-$\tor$-2}]
{\Gamma \ts T_1 \wf ~~~~ \Gamma \ts T_2 \wf}
{\Gamma \ts T_2 \sub T_1 \tor T_2}

\end{multicols}

\vspace{0.1em}

\begin{multicols}{2}

\infrule[\textsc{$\sub$-trans}]
{\Gamma \ts T_1 \sub T_2 \\ \Gamma \ts T_2 \sub T_3}
{\Gamma \ts T_1 \sub T_3}

\infrule[\textsc{$\sub$-hyp}]
{(x: T) \in \Gamma ~~~~  \Gamma \ts T \wf \\
 \Gamma \ts T \ni (\Ldecl L S U) \\
 \Gamma \ts S \wf ~~~~  \Gamma \ts U \wf}
{\Gamma \ts S \sub U}

\end{multicols}

\vspace{0.1em}

\newruletrue

\begin{multicols}{2}

\infrule[\textsc{$\sub$-self-l}]
{(x: T) \in \Gamma \\  \Gamma \ts T \wf}
{\Gamma \ts x\sing \sub T}

\infrule[\textsc{$\sub$-ex-l}]
{\Gamma \envplus{x: S} \ts T \sub U \\
 \Gamma \ts U \wf ~~~~ \Gamma \envplus{x: S} \ts S \wf}
{\Gamma \ts {\exTyp x S T} \sub U}

\infrule[\textsc{$\sub$-self-r}]
{(x: T) \in \Gamma \\
 \Gamma \ts x\sing \wf \\
 \Gamma \ts y\sing \wf}
{\Gamma \ts y\sing \sub x\sing}

\infrule[\textsc{$\sub$-ex-r}]
{(x: S') \in \Gamma ~~~~ \Gamma \ts S' \sub S \\
 \Gamma \ts T \sub U}
{\Gamma \ts T \sub {\exTyp x S U}}

\end{multicols}

\newrulefalse

\linesep

\begin{multicols}{2}[\judgement{Declaration subtyping}{\fbox{$\Gamma \ts D \sub D'$}}]

    \infrule[\textsc{subdec-typ}]
            {~~\Gamma \ts S' \sub S ~~~~ \Gamma \ts T \sub T'~~}
            {\Gamma \ts (\Ldecl L S T) \sub (\Ldecl L {S'} {T'})}

    \infrule[\textsc{subdec-mtd}]
            {\Gamma \ts S' \sub S ~~~~ \Gamma \ts T \sub T'}
            {\Gamma \ts (\mdecl m S T) \sub (\mdecl m {S'} {T'})}

\end{multicols}
}
\end{figure}


%%%%%%%%%%%%%%%%%%%%%

\begin{figure}
\figurebox{
\pdfbookmark[0]{Typing}{typing}
\begin{multicols}{2}[\judgement{Term Typing}{\fbox{$\Gamma \ts t : T$}}]

\infrule[\textsc{ty-var}]
{(x: T) \in \Gamma\\
 \Gamma \ts T \wf}
{\Gamma \ts x: T}

\infrule[\textsc{ty-call}]
{\Gamma \ts t : T ~~ \Gamma \ts u : U  ~~ \Gamma \ts V \wf \\
 \Gamma \ts T \ni (\mdecl m U V)}
{\Gamma \ts {\trmcall t m u} : V}

\infrule[\textsc{ty-new}]
{\Gamma \envplus{x: T} \ts \{\seq{d}\} : T \\
 \Gamma \envplus{x: T} \ts u : U ~~~~ \Gamma \ts U \wf}
{\Gamma \ts {\trmnew x {\seq{d}} u}: U}

\infrule[\textsc{ty-sbsm}]
{\Gamma \ts t: T_1\\
 \Gamma \ts T_1 \sub T_2}
{\Gamma \ts t: T_2}

\end{multicols}

\linesep

\begin{multicols}{2}[\judgement{Initialization Typing}{\fbox{$\Gamma \ts d : D$}}]

\infrule[\textsc{ty-tdef}]
{\Gamma \ts T \wf}
{\Gamma \ts (\deftyp L T) : (\Ldecl L T T)}

\infrule[\textsc{ty-mdef}]
{\Gamma \ts T \wf ~~~~ \Gamma \ts U \wf \\
 \Gamma \envplus{x: T} \ts u : U}
{\Gamma \ts (\defmtd m x T U u) : (\mdecl m T U) }

\end{multicols}

\linesep

\begin{multicols}{2}[\judgement{Initialization List Typing}{\fbox{$\Gamma \ts \{\seq{d}\} : T$}}]

\infrule[\textsc{ty-nil}]
{}
{\Gamma \ts \{\} : \Top}

\infrule[\textsc{ty-cons}]
{ label(d) \notin labels(\seq{d})\\
  \Gamma \ts \{ \seq{d} \} : T ~~~~ \Gamma \ts d : D }
{\Gamma \ts \{ \seq{d}, d \} : T \tand \{ D \} }

\end{multicols}
}

\end{figure}

%%%%%%%%%%%%%%%%%%%%%%%%%%%%%%%%%


\end{document}
